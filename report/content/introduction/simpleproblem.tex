\section{A Simpler Problem}

To make the problem of finding the fastest path between two points on a road network simpler, 
we assume the speed of the electric vehicle is constant and we assume the charging time at
charge stations is constant. 

\subsection{The graph representation}
We represent the road network as a graph where edges represents the roads and vertices represents
road intersections or charge stations. An edge is a double of time and cost:

\[E = (\textit{TIME}, \textit{COST}) \]

$\textit{TIME}$ represents the time it takes to pass an edge. $\textit{TIME}$ is the distance of
the road divided by the speed, but as mentioned earlier, for this simpler problem we assume
speed is constant, so $\textit{TIME}$ is given solely by the distance of the road. $\textit{COST}$ 
represents the energy cost for the an edge. $\textit{COST}$ is the energy usage per distance 
times the distance. Since the energy usage per distance is also given by the speed, this value
will be constant and so $\textit{COST}$ is solely given by the distance of the edge.

A vertex is a single of time: 
 
\[V = (\textit{TIME})\]

$\textit{TIME}$ is the time it takes to charge an electric vehicle one unit of energy e.g. minutes 
per kWh. The time it take to charge an electric vehicle one unit of energy depends on the current
capacity of the electric vehicle, but as mentioned earlier, for simplicity, we assume charging 
time is constant no matter the current capacity of the battery.

\subsection{A simple example}

We want to give a simple example of how an algorithm might work, for finding the fastest path 
between two points when the charging time is constant and the speed is constant. The example is
illustrated in \ref{fig:simpleexample}. We want to find the fastest path from $s$ to $t$. The
electric vehicle used has a battery capacity of 50 kWh. The unit of $\textit{TIME}$ for vertices
is minutes/kWh, the unit of $\textit{TIME}$ for edges is minutes and the unit of $\textit{COST}$
for edges is kWh. E.g. the edge (s, a) takes 10 minutes and costs 10 kWh to pass, and the charge
station at vertex a charges with a speed of 3 minutes/kWh. Vertices with the value $\infty$, are
considered intersections of roads and they are without charging stations.

   

\begin{figure}
\centering
\begin{tikzpicture}
  [scale=.4,>=stealth',shorten >=1pt,auto,node distance=2.5cm,
  thick,main node/.style={circle,fill=blue!20,draw,font=\sffamily\scriptsize\bfseries}]
  \node[main node] (1) {s};
  \node[main node] (2) [below right of=1] {a (3)};
  \node[main node] (3) [above right of=2] {b ($\infty$)};
  \node[main node] (4) [below right of=3] {c (2)};
  \node[main node] (5) [above right of=4] {d ($\infty$)};
  \node[main node] (6) [below right of=5] {e (4)};
  \node[main node] (7) [above right of=6] {f ($\infty$)};
  \node[main node] (8) [below right of=7] {g (2)};
  \node[main node] (9) [above right of=8] {t};

  \path[every node/.style={font=\sffamily\tiny}] 
  (1) edge node [left] {(10, -10)} (2)
  (2) edge node [left] {(10, -10)} (3)
  (3) edge node [left] {(20, -20)} (4)
  (4) edge node [left] {(20, -20)} (5)
  (5) edge node [left] {(30, -30)} (6)
  (6) edge node [left] {(10, -10)} (7)
  (7) edge node [left] {(40, -40)} (8)
  (8) edge node [left] {(10, -10)} (9);
\end{tikzpicture}
\caption{Figure: Illustration of a simple road network with start point $s$ and end point $t$}
\label{fig:simpleexample}
\end{figure}

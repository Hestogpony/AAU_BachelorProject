\section{A Simpler Problem}

To make the problem of finding the fastest path between two points on a road network simpler, 
we assume the speed of the electric vehicle is constant and we assume the charging time at
charge stations is constant. 

\subsection{The graph representation}
We represent the road network as a graph where edges represents the roads and vertices represents
road intersections or charge stations. An edge is a double of time and cost:

\[E = \{\textit{TIME}, \textit{COST}\} \]

$\textit{TIME}$ represents the time it takes to pass an edge. $\textit{TIME}$ is the distance of
the road divided by the speed, but as mentioned earlier, for this simpler problem we assume
speed is constant, so $\textit{TIME}$ is given solely by the distance of the road. $\textit{COST}$ 
represents the energy cost for the an edge. $\textit{COST}$ is the energy usage per distance 
times the distance. Since the energy usage per distance is also given by the speed, this value
will be constant and so $\textit{COST}$ is solely given by the distance of the edge.

A vertex is a single of time: 
 
\[V = \{\textit{TIME}\} \]

$\textit{TIME}$ is the time it takes to charge an electric vehicle one unit of energy e.g. minutes 
per kWh. The time it take to charge an electric vehicle one unit of energy depends on the current
capacity of the electric vehicle, but as mentioned earlier, for simplicity, we assume charging 
time is constant no matter the current capacity of the battery.

\subsection{A simple example}

We want to give a simple example of how an algorithm might work, for finding the fastest path 
between two points when the charging time at charge stations is constant and the speed of the 
vehicle is constant. A graph, representing a road-network is illustrated in \ref{fig:simpleexample}. 
We want to find the fastest path from $s$ to $t$. The electric vehicle used has a battery capacity 
of 50 kWh. The unit of $\textit{TIME}$ for vertices is minutes/kWh, the unit of $\textit{TIME}$ 
for edges is minutes and the unit of $\textit{COST}$ for edges is kWh. E.g. the edge (s, a) takes 
10 minutes and costs 10 kWh to pass, and the charge station at vertex a charges with a speed of 
3 minutes/kWh. Vertices with the value $\infty$, are considered intersections of roads and they are 
without charging stations.  

The Algorithm for finding the fastest path is called \texttt{FASTEST-PATH} and it's given in \ref{alg:fastestpath}.
The main while loop runs until the current node is $t$, since this is where we want to get to. The
$\textnormal{currentNode}$ variable always contains the node we are currently at. The first if-statement checks 
whether our battery is full. If this is the case then we will always drive. The else if-statement checks if there 
is a faster charge station reachable from the node we are currently at with our current battery capacity
($\textnormal{curEnergy}$). If this is the case we will drive to this charge station instead. The else-statement
checks whether which charge stations is the best, if we had a full battery capacity ($\textnormal{v.maxBat}$) and
we then charge just enough to reach this charge station.

There are several functions inside \texttt{FASTEST-PATH}, namely: \texttt{REACHABLES}, \texttt{PATHLENGTH} and 
\texttt{BCS}. 

\texttt{REACHABLES} takes a path, $P$, a vehicle, $\textnormal{v}$, and a battery capacity, 
$\textnormal{v.energy}$ as input and it output a list of nodes, $\textnormal{nodes}$, which we can reach from 
the node we are currently at. 

\texttt{PATHLENGTH} takes a path, $P$ as input and outputs the time it takes to get through this path.

\texttt{BCS} takes a list of nodes as input and outputs the node which is the fastest charges station. 
  
\begin{algorithm}
\begin{algorithmic}[H]
\Procedure{Fastest-Path}{$P[0...n], \textnormal{v}, \textnormal{s}, \textnormal{t}$}
	\State $\textnormal{driveTime} \gets \textnormal{PATHLENGTH(P[0...n])}$
	\State $\textnormal{chargeTime} \gets 0$	
	\State $\textnormal{currentNode} \gets \textnormal{s}$
  \State $\textnormal{curEnergy} \gets \textnormal{v.curBat}$
	\While{$\textnormal{currentNode} \neq \textnormal{t}$}
    \State $\textnormal{bestNode} \gets \textnormal{BCS}(\textnormal{REACHABLES}(P[\textnormal{currentNode}...n], \textnormal{v},\textnormal{curEnergy}))$
		\If {$\textnormal{curEnergy} = \textnormal{v.maxBat}$}
      \State $\textnormal{curEnergy} \gets \textnormal{curEnergy} - (\textnormal{PATHLENGTH}(P[\textnormal{currentNode}...\textnormal{bestNode})/\textnormal{v.kmPerKWh})$
			\State $\textnormal{currentNode} \gets \textnormal{bestNode}$

		\ElsIf {$\textnormal{currentNode.speed} < \textnormal{bestNode.speed}$}
      \State $\textnormal{curEnergy} \gets \textnormal{curEnergy} - (\textnormal{PATHLENGTH}(P[\textnormal{currentNode}...\textnormal{bestNode})/\textnormal{v.kmPerKWh})$
			\State $\textnormal{currentNode} \gets \textnormal{bestNode}$
		\Else
			\State $\textnormal{chargeNode} \gets \textnormal{BCS}(
					\textnormal{REACHABLES}(P[\textnormal{currentNode}...n], \textnormal{v}, \textnormal{v.maxBat})$
			\State $\textnormal{kWhToCharge} \gets (P[\textnormal{currentNode,...,chargeNode}] / \textnormal{v.kmPerKWh}) - \textnormal{curEnergy}$
			\State $\textnormal{chargeTime} \gets \textnormal{chargeTime} + (\textnormal{kWhToCharge} / \textnormal{chargeNode.speed}$
		\EndIf
	\EndWhile
	\State \textbf{Return} $\textnormal{driveTime} + \textnormal{chargeTime}$
\EndProcedure
\end{algorithmic}
\label{alg:fastestpath}
\end{algorithm}

\begin{algorithm}
\begin{algorithmic}[1]
\Procedure{reachables}{$P[0...n], \textnormal{v}, \textnormal{energy}$}	
	\For{$x \gets 0$ to $n-1$}
		\State $\textnormal{tempDistance} \gets Edge(P[x], P[x+1]).\textnormal{distance}$
		\If {$(\textnormal{tempDistance} / \textnormal{v.kmPerKWh}) < \textnormal{energy}$}
			\State $\textnormal{nodes.add}(P[x+1])$
		\EndIf	
	\EndFor
	\State \textbf{Return} $\textnormal{nodes}$
\EndProcedure
\end{algorithmic}
\end{algorithm}

\begin{algorithm}
\begin{algorithmic}[1]
\Procedure{pathLength}{$P[0...n]$} 
  \State $\textnormal{travelTime} \gets 0$ 
  \For{$x \gets 0$ to $n-1$}
    \State $\textnormal{travelTime}$ $\gets \textnormal{travelTime} + Edge(P[x], P[x+1]).\textnormal{distance} / Edge(P[x], P[x+1]).\textnormal{speedLimit}$
  \EndFor
  \State \textbf{Return} $\textnormal{travelTime}$
\EndProcedure
\end{algorithmic}
\end{algorithm}




\begin{figure}
\centering
\begin{tikzpicture}
  [scale=.4,>=stealth',shorten >=1pt,auto,node distance=2.5cm,
  thick,main node/.style={circle,fill=blue!20,draw,font=\sffamily\scriptsize\bfseries}]
  \node[main node] (1) {s};
  \node[main node] (2) [below right of=1] {a (3)};
  \node[main node] (3) [above right of=2] {b (0)};
  \node[main node] (4) [below right of=3] {c (2)};
  \node[main node] (5) [above right of=4] {d (0)};
  \node[main node] (6) [below right of=5] {e (4)};
  \node[main node] (7) [above right of=6] {f (0)};
  \node[main node] (8) [below right of=7] {g (2)};
  \node[main node] (9) [above right of=8] {t ($\infty$)};

  \path[every node/.style={font=\sffamily\tiny}] 
  (1) edge node [left] {(10, -10)} (2)
  (2) edge node [left] {(5, -5)} (3)
  (3) edge node [left] {(20, -20)} (4)
  (4) edge node [left] {(15, -15)} (5)
  (5) edge node [left] {(20, -20)} (6)
  (6) edge node [left] {(10, -10)} (7)
  (7) edge node [left] {(15, -15)} (8)
  (8) edge node [left] {(10, -10)} (9);
\end{tikzpicture}
\caption{Figure: Illustration of a simple road network with start point $s$ and end point $t$}
\label{fig:simpleexample}
\end{figure}

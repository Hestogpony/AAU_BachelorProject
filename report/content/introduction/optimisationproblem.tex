\section{The Optimisation problem}

We intend to find the fastest path between two points, A and B, for electric vehicles. To find the 
fastest path we need to measure time. Time can be spend on two things: driving on the road or on charging
at a charging station if necessary. Since the charging times can vary a lot from charging station to charging station 
perhaps it would be fastest to take a slower path if it had a faster charging station on it. Whether one path is 
faster than another can be formulated as an optimisation problem:

\[\displaystyle \sum_{i=1}^{n} \textit{Drivetime}[i] * k \displaystyle \sum_{j=1}^{n} \textit{Chargetime}[j]\]

Where $\textit{Drivetime}[i]$ is the time it takes to pass road $i$, $k$ is an uncertainty constant,
which we use because it is difficult to keep a constant speed. Finally, $\textit{Chargetime}[j]$ is the time
spend charging on charge station $j$.

%Min(Time(speed)) = Min(TimeSpendDriving(speed) + TimeSpendCharing(speed)) = Min
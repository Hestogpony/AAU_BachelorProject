
\section{Electrical Vehicles}

We investigate the range, charging options, charging times, battery capacity and other factors 
which may be relevant when calculating the fastest path between two points for electric vehicles.
We focus on two electric vehicles: the Nissan Leaf and Tesla's S model. The reasons why we have 
chosen these two cars in particular is the fact that both of the cars, today, are among the most 
popular electric vehicles. 

\subsection{The Range of electric vehicles}

The range of an electric vehicles depends on many parameters, such as: the battery capacity, how efficient the battery
is used, the speed which is driven at, the drag of the car, the wind resistance, how much air conditioning is 
used etc. The two most important factors, though, are the battery capacity and the speed which is driven at. 
Tesla's S model comes with different battery sizes. We choose to make our calculation for the Tesla S model with a 
battery size of 85 kWh which is the biggest battery size of any electric vehicle right now. The Nissan Leaf on the 
other hand comes with a 24 kWh battery.\\

The range of the Tesla S model with 85 kWh battery is approximately given by the following function:

\[f(x, y, z) = \frac{x}{y * (4z + 49)}\]

The range of the Nissan Leaf given speed is approximately given by the following function:

\[f(x, y, z) = \frac{x}{y * ((\textit{unknown})z + (\textit{unknown}))}\]

$x$ is the current battery capacity in watt-hours, $y$ is the the energy usage in watt-hours per kilometres and $z$ is the 
speed in kilometres per hour. In \ref{table:rangegivenspeedtesla} calculations for the range of Tesla S Model is given. 
As seen in the table the range decreases exponentially with the speed.\\

\begin{table}
	\begin{center}
    	\begin{tabular}{ | >{\centering\arraybackslash}m{1in} | >{\centering\arraybackslash}m{1in} | >{\centering\arraybackslash}m{1in} | }
    	\hline
    	\textbf{Speed (km/t)}  	& \textbf{Usage (kWh/km)} 	& \textbf{Range (km)} 	\\ \hline
    	30 				& 0.169				& 502.96  		\\ \hline
    	55 				& 0.269				& 315.99		\\ \hline
    	80 				& 0.369				& 230.35		\\ \hline
    	130				& 0.569 			& 149.38		\\ 
    	\hline    
    	\end{tabular}
    	\end{center}
    	\caption{Calculates the range of Tesla's S Model with battery size of 85 kWh, given the speed}
		\label{table:rangegivenspeedtesla}
\end{table}

If we were to drive from A to B, where the distance between A and B were 500 km, in a Tesla S model. 
At which speed would we arrive to B fastest: at 30, 55, 80 or 139 km/t? For simplicity we assume there are
infinitely many charging stations on the road, so we can charge as soon as our battery hits 0 \% capacity.
The time for charging from 0 \% to 100 \% takes 90 min at each charging station. 
The result is calculated in \ref{table:500kmfastesttesla}. As seen in the table it is fastest to drive 500 km
with a speed of 130 km/t. If the charging times were even longer perhaps it would be faster to drive 80 km/t 
instead. Also, on a regular road there wouldn't be infinitely many charging stations, so in some cases we would
spend time looking for a charging station or we would have to charge even though the battery wasn't empty, 
because the next charging station was very far away. 

\begin{table}
	\begin{center}
    	\begin{tabular}{ | >{\centering\arraybackslash}m{1in} | >{\centering\arraybackslash}m{1in} | 
    					   >{\centering\arraybackslash}m{1in} | >{\centering\arraybackslash}m{1in} |}
    	\hline
    	\textbf{Speed (km/t)}  	& \textbf{Time driving (min)} 	& \textbf{Time charging (min)}	& \textbf{Total time (min)} \\ \hline
    	30 				& 1000					& 0  					& 1000				\\ \hline
    	55 				& 545					& 90					& 635				\\ \hline
    	80 				& 375					& 180					& 555				\\ \hline
    	130				& 231 					& 270					& 501				\\ 
    	\hline    
    	\end{tabular}
    	\end{center}
    	\caption{Calculation of which speed is the optimal for driving 500 km fastest}
		\label{table:500kmfastesttesla}
\end{table}

\subsection{The Charging time of electric vehicles}
  
There are different ways of getting an electric vehicles to a full charge. One way is to change the
flat battery with a charged battery at a battery changing station.  Another way is to charge the flat 
battery at home, at work or at a recharging station. Changing the flat battery to a charged one, can 
take down to a couple of minutes at a battery changing station. The battery charging time depends on
the capacity of the battery. The bigger capacity the battery has, the longer it's going to take to 
recharge it. Also, the more full the battery is, the longer time it takes to charge it more. For example,
it takes a lot longer time to charge the last 20 \% of the battery than it takes to charge the first
20 \% of the battery.

\subsubsection{Charging plug-in standards}


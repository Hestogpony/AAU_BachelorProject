\section{Algorithms}\label{sec:algo}
Now that we are able to solve a path optimally but also heuristically, we are ready to describe a full algorithm for solving the fastest path problem. 

An approach for solving the full problem, for the whole road network would be to iterate over every single simple path(a path without cycles) in the graph and figure out the best way to travel that path according to either the travel\_time procedure described in Section \ref{sec:greedy} or the Linear Programming model described in Section \ref{sec:LP}. However, doing this would be way too complex as the number of simple paths in a graph can be as much as $O(n!)$ which is too much.

Instead, one can utilize existing graph algorithms for reducing the number of paths being looked at. e.g. in Dijkstra's algorithm, the number of paths is substantially reduced by relaxing pairs of nodes, effectively reducing the number of paths substantially. This is possible, because the shortest path problem has an optimal substructure. Unfortunately, the fastest path problem does not have such a substructure.

Along with the travel\_time procedure, which heuristically solve an edge local optimally, we can now utilize Dijkstra's algorithm for solving the fastest path problem. Instead of relaxing on the distance of edges, we relax on the time being calculated in the travel\_time procedure. Note that the Linear Programming could theoretically also be applied here. Though it works on a full path one can apply linear programming for each edge along with the predecessor chain, thus solving intermediate paths in a local-optimal manner. This will certainly not give the global optimal solution, but obviously better than using travel\_time.
\begin{algorithmic}
\Function{GreedyHeuristic}{$RN,s,t,EV$}
    \ForAll{$v \in RN.V$} 
        \State $v.time = \infty$
        \State $v.predecessor = NIL$
        \State $v.preCS = NIL$
        \State $v.B_{cur} = 0$
    \EndFor
    \State $s.time = 0$
    \State $s.B_{cur} = EV.B_{cur}$

    \State $Q = PriorityQueue$
    \State $insert(Q, (s.time, s))$ 
    \While{$Q \neq \emptyset$} 
        \State $u = extractmin(Q)$
        \ForAll{$v \in RN.adj(u)$} 
            \State $time,preCS,B_{cur},energy = $
            \State $travel\_time(RN, u, v, EV)$
            \If{$v.time > u.time + time$} 
                \State $v.time = u.time + time$
                \State $v.predecessor = u$
                \State $v.B_{cur} = B_{cur}$
                \State $v.preCS = preCS$
                \State $insert(Q, (v.time, v))$ 
            \EndIf
            \If{ $preCS \neq NIL$}
                \State $a$
            \EndIf
        \EndFor
    \EndWhile
    \State \Return $t.time, t.path$
\EndFunction
\end{algorithmic}\label{alg:fastest_path}


The above pseudo code, uses the greedy heuristic travel\_time procedure to solve the fastest path problem, it resembles the structure of Dijkstra's algorithm. The PriorityQueue uses Fibonacci-heap, which is excellent for sparse graphs such as the road network. In the relaxation, if a neighbour achieves a better time, we wish to update the time, predecessor, previous charge stations and insert this node into the min-heap.\todo[inline]{missing part of algorithm, cant describe more}

The worst case running time of the greedy heuristic algorithm is bound by three procedures. First there is the main procedure that is derived directly from Dijkstra's, which has a worst case complexity of $O(|E|+|V|\log|V|)$, with the Fibonacci heap. To that we need to multiply $O(|V|)$ from the travel\_time procedure, as this procedure might go through all vertices, if the vehicle is set to charge at a new charging station for every found edge and every vertex in the graph is also a charging station. Finally we have the procedure that maintain a set of charging stations relevant on each vertex. In the special case where the vehicle have an infinite battery capacity, but start without any battery, and every vertex found happens to be a charging station with a charge rate in a decreasing order, the procedure would have to multiply $O(|V|)$ in the worst case.
However only one of the last two procedure's $O(|V|)$ complexity is added, as the case where the algorithm chose to charge at a new charging station for every edge, never happens with an infinite battery capacity. Likewise the case where the procedure maintains a list of every charging station, never happens if the car has a limited battery capacity. Resulting in a worst case running time of $O(|V||E|+|V^2|\log|V|)$. %http://www.wolframalpha.com/input/?i=v%28e%2Bv*log%28v%29%29

In practice Dijkstra's algorithm run faster on a sparse graph, which road networks are, and both of the remaining procedures are simultaneously limited by the fact that only a small amount of charging stations exist and that the EV battery capacity is limited.
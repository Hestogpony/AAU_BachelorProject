\section{Introduction}

As the adoption rate of electric vehicles increases \cite{ev-sales-increasing}, the need for a better and more complex route planing solution appears. Electric vehicles generally have a shorter range as well as a significantly slower recharge time, compared to traditional gasoline cars. As the recharge time have become a significant factor in the total travel time, the time spend recharging should be taken into account by a route planning system, to accommodate for an otherwise large margin of error. Time spend recharging batteries is inherently variable, as the charge rate is affected by both the charge station's charge rate as well as the battery's current charge. When planning a fastest-path route for an electric vehicle, it is therefore paramount to be able to incorporate charge station's charge rates and electric vehicle's battery.\\

Another important aspect is the electric vehicle's energy consumption rate. With the now significant time spend recharging, a high consumption rate while driving, can in turn result in a slow route due to increased charging time. The consumption rate of a electric vehicles generally increases polynomially with speed, primarily due to aerodynamics \footnote{Aerodynamic losses are important especially at high speeds. The force of air friction is given by equation: $F = \frac{1}{2} \rho V^2 A C_d$, where $\rho$ is the air density, $A$ is the frontal area of the vehicle, $C_d$ is the drag coefficient and $V$ is the speed}, so the consumption rate should also be taken into account by the route planning system. Furthermore the rate of charging of the battery in electrical vehicles is a polynomial function, i.e. the battery is charged substantially faster from 0\% to 80\% compared with 80\% to 100\%. We will ignore this fact, as this is arguably not a substantial problem. One could simply linearize the function for the battery.\\

It should be clear that no traditional shortest path algorithm is able to accommodate for the variables and relationships between the speed, the consumption rate, the charge rate and the current charge of the electric vehicle. In this article, we present a Greedy-heuristic algorithm for solving the problem of optimising a route plan for electric vehicles. We also introduce a linear programming approach which could be combined with the Greedy-heuristic algorithm. The algorithm with and without linear programming are analysed on their running time and reliability.





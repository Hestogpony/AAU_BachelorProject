\section{Introduction}

As the adoption rate of electric vehicles increases \cite{Henry2013} and the charging station infrastructure is still limited, the need for a route planning solution that incorporates charging stations and recharge time emerges. EVs generally have a short range, generally ranging between 100km and 250km. Though some EV models like Tesla's S-models can reach up to $400\si{km}$ on a single charge \cite{teslacon}. Charging stations require a significantly longer time to refill a battery, compared to refuelling a gasoline car. Splitting charging stations into two groups of fast and slow charge rates, where the less widespread fast charging stations takes between 30 minutes and 2 hours to fill a battery and the more widespread slow charging station takes between 4 and 12 hours. Globally, there exist about 50.000 slow charging stations and 2000 fast charging stations\cite{Globalevoutlook}. As the time spend recharging is a significant factor in terms of the total travel time, a route planning system for EVs should take charging time and driving time into account.

Another central aspect is the EVs energy consumption rate. With slow recharging, a high consumption rate while driving fast can in turn result in a combined slower route, due to increased charging needed. The energy consumption rate of EVs rises polynomially with speed due to air aerodynamics\footnote{Aerodynamics play a big role in the energy consumption for vehicles, the energy losses increase greatly at high speeds. The force of air friction is given by this equation: $F = \frac{1}{2} \rho V^2 A C_d$, where $\rho$ is the air density, $A$ is the frontal area of the vehicle, $C_d$ is the drag coefficient and $V$ is the speed}.

It should be clear that no traditional shortest path algorithm is able to accommodate for the relationships between the speed, consumption rate, charge rate and current battery of the electric vehicle. 

In Section \ref{sec:notation} we describe the mathematical notation and model of this paper and present the definition of the fastest path problem. Further more, we give a brief example of why a shorter path does not necessarily imply a faster path for EVs. In Section \ref{sec:relatedwork} we present related work concerning e.g.\ route planning, constrained shortest path problems etc. In Section \ref{sec:fastestpath} we define the problem of optimising a simple path (a path without cycles) and present two possible solutions for solving this problem. One solution approximates an optimal solution using linear programming (LP) and the other solution uses greedy choices and heuristics. In Section \ref{sec:algo} we present a greedy heuristic algorithm for solving an entire road network. In section \ref{sec:experiments} we present the experiments used for performance evaluation of the greedy heuristic solution. A naive algorithm was implemented for performance comparison in the experiments. Finally, in Section \ref{sec:conclusion} we present the conclusion of the paper.    
\section{Introduction}

As the adoption rate of Electric Vehicles (EV) increases \cite{Henry2013} and because the charging station infrastructure is poor, the need for a route planning system that incorporates charging stations and recharge time emerges. EVs have a short range, generally between $100\si{km}$ and $250\si{km}$. Though some EV models like Tesla's Model S can reach up to $426\si{km}$ on a single charge \cite{teslacon}. EVs require a significantly longer time to recharge, compared to refuelling a gasoline car. Charging stations can be split into two groups: the less widespread "fast charging" stations takes between 30 minutes and 2 hours to recharge an EV and the more widespread "slow charging" stations takes between 4 and 12 hours. Globally, there exist about 50.000 slow charging stations and 2.000 fast charging stations\cite{Globalevoutlook}. As the time spent recharging is a significant factor in terms of the total travel time, a route planning system for EVs should take charging time and driving time into account.

A central aspect is the EVs energy consumption rate. Slow recharging combined with a high consumption rate while driving fast, can in turn result in a combined slower route, due to increased charge time. The energy consumption rate of vehicles increases polynomially with speed due to aerodynamics\footnote{Aerodynamics play a big role in the energy consumption for vehicles. The energy losses increase greatly at high speeds. The force of air friction is given by the following equation: $F = \frac{1}{2} \rho V^2 A C_d$, where $\rho$ is the air density, $A$ is the frontal area of the vehicle, $C_d$ is the drag coefficient and $V$ is the speed}.

It should be clear that no traditional shortest path algorithm is able to accommodate for the relationships between the speed, consumption rate, charge rate and current battery of the electric vehicle. 

In Section \ref{sec:notation} we describe the mathematical notation and model of this paper and present the definition of the fastest path problem. Furthermore, we give a brief example of why a shorter path does not necessarily imply a faster path for EVs. In Section \ref{sec:relatedwork} we present the related work. In Section \ref{sec:fastestpath} we define the problem of optimising a simple path (a path without cycles) and present two possible solutions for solving this problem. One solution approximates an optimal solution using Linear Programming (LP) and the other solution uses greedy choices and heuristics. In Section \ref{sec:algo} we present a greedy heuristic algorithm for solving an entire road network. In Section \ref{sec:experiments} we present the experiments used for evaluation of the greedy heuristic solution. A naive algorithm was implemented for comparison in the experiments. Finally, in Section \ref{sec:conclusion} we present the conclusion of this paper.    


\section{Introduction}

As the adoption rate of electric vehicles (EV) increases \cite{Henry2013} and the charging station infrastructure is still limited, the need for a route planing solution that incorporates charging stations and recharge time emerges. EVs generally have a short range, where premium models can reach upward to 500km on a single charge, the normal range is between 100km and 250km. Charging stations need a significantly longer time to refill a battery, compared to traditional gasoline stations. Splitting charging stations into two groups of fast and slow charge rates, where the less widespread fast charging will need between 30 minutes and 2 hours to fill a battery and the more widespread slow-charging station need between 4 and 12 hours. Globally there exist about 50.000 slow charging stations and 2000 fast charging stations\cite{Globalevoutlook}. As the time spend recharging is a significant factor in terms of the total travel time, a route planning system should take charge rates and the driving distance to the charge stations into account.

Another central aspect is the EVs energy consumption rate. With slow recharging, a high consumption rate while driving fast can in turn result in a combined slower route, due to increased charging needed. The energy consumption rate of EVs increases polynomially with speed due to air aerodynamics\footnote{Aerodynamic losses are important, especially at high speeds. The force of air friction is given by equation: $F = \frac{1}{2} \rho V^2 A C_d$, where $\rho$ is the air density, $A$ is the frontal area of the vehicle, $C_d$ is the drag coefficient and $V$ is the speed}.

It should be clear that no traditional shortest path algorithm is able to accommodate for the described variables and relationships between the speed, the consumption rate, the charge rate and the current charge of the electric vehicle. In this paper, we present a greedy heuristic algorithm for solving the problem of optimising a route plan for EVs. \todo[inline]{Introduce the greedy heuristic method shortly} We also introduce a linear programming approach which could be combined with the greedy heuristic algorithm to increase path finding abilities at the cost of the running time.  

\section{Introduction}
As the adoption rate of electric vehicles increases, the need for a better and more complex route planing solution appears. Electrical vehicles generally have a shorter range as well as a significantly slower recharge time, compared to traditional gasoline cars. As the recharge time have become a significant factor in the total travel time, the time spend recharging should be taken into account by a route planning system, to accommodate for an otherwise large margin of error. Time spend recharging batteries is inherently variable, as the charge rate is affected by both the charge station's charge rate as well as the battery's current charge. When planing a fastest-path route for an electric vehicle, it is therefore paramount to be able to incorporate charge station's charge rates and electric vehicle's battery.

Another important aspect is the electric vehicle's energy consumption. With the now significant time spend recharging, a high energy consumption rate while driving, can in turn result in a slower route. The energy consumption of electrical vehicles generally increases polynomially with speed so this should also be taken into account by the route planning system.

It should be clear that no traditional shortest path algorithm is able to accommodate for the variables and relationships between the travel speed, the charge rate and the current charge of the electrical vehicle. In this article we present two different approaches to solving the problem of optimising a route plan for electrical vehicles. One is a linear program..... The other is naive solution which uses heuristics..... Both approaches are analysed on their running time and reliability, based on a real world road network as input.




\begin{abstract}

We describe a greedy heuristic route planning algorithm for finding time optimised paths between two points in a
large road network graph, for electric vehicles (EVs). We further introduce a linear programming 
approach, which could be combined with the greedy heuristic algorithm and likely improve route planning abilities 
at the cost of run time performance. Unlike many of today's GPS systems, the greedy heuristic 
algorithm takes both the drive time and the recharge time into account. Since the recharging possibilities for 
electric vehicles are still limited on Danish roads and the recharging process takes a long time compared to refuelling 
times of regular gasoline cars, this is necessary in order to give the electric vehicles users a 
useful route plan.

We compared the performance of the greedy heuristic algorithm with a naive algorithm on road
networks extracted from OpenStreetMap (OSM). We modified the OSM road network, so it contained Danish speed limits
and charging stations at variable positions and with variable charge rates. We compared the performance of the algorithms 
with four different parameters: the density of charging stations in the road network, the charge rates of the charging stations 
in the road network, the size of the road network measured in distance and the complexity of the road network measured 
in number of nodes. HOW DID IT PERFORM?!      

%We describe a greedy heuristic  algorithm for electric vehicles, this algorithm uses parameters about both the road network and the vehicle itself, to find a time efficient route plan between two points. We further introduce a linear programming method, which combined with the greedy-heuristic algorithm, can improve effectiveness and efficiency at the cost of computation time. Unlike the most common of today's navigation systems, the presented greedy-heuristic algorithm takes both the driving time and recharge time into account. This is a necessity as the recharging infrastructure is still limited, both in terms of the number of charge stations and their charge rates.\\


\end{abstract}
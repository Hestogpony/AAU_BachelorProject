\subsection{Optimisation Problem}\label{sec:optiprob}
This section introduces a formal description of how to solve a simple path of length $n$ optimally.

\begin{figure}[!htb]
\centering
    \begin{tikzpicture}[shorten >=1pt,node distance=2.5cm,>=stealth',thick]
        \node[state] (1) {$u_1$};
        \node[state] (2) [right of=1] {$u_2$};
        \node[] (dots) [right of=2] {$\dots$};
        \node[state] (n) [right of=dots] {$u_n$};
        \node[state,draw=none] (d1) [right of=n] {};
        \draw [->] (1) to[right] node[auto] {$e_1$} (2);
        \draw [->] (2) to[right] node[auto] {$e_2$} (dots);
        \draw [->] (dots) to[right] node[auto] {$e_{n-1}$} (n);
        %\draw [->] (n) to[right] node[auto] {$e_n$} (d1);
    \end{tikzpicture}
    \caption{Example of a path consisting of $n$ edges} \label{fig:pathexample}
\end{figure}

The objective of the optimisation problem is to minimise the time spent driving and the time spent charging on a path. The total time spent on a path can be expressed as the following objective function:
\begin{equation*}
\begin{aligned} &
{\text{min:}}
& & \sum_{i=1}^{n-1} \left(\frac{D(e_i)}{v_{e_i}} + CT_{u_i} \right)\\
\end{aligned}
\end{equation*}
Where $D(e_i)$ is the length of road segment $i$, $v_{e_i}$ is the speed driven on road segment $i$ and $CT_{u_i}$ is the charge time at the $i^{\text{th}}$ vertex of the path illustrated in Figure \ref{fig:pathexample}. It should be clear that the objective function of this problem is concerned with time because \( D(e_i)/v_{e_i} \) is the time spent driving and $CT_{u_i}$ is the charge time. The objective function has two unknown variables: $v_{e_i}$ and $CT_{u_i}$. The unknown variables of the objective function are constrained by the physical properties of the $EV$ driving the path. We will now ensure that the physical constraints defines the feasible region of the optimisation problem.

On every edge the speed of the EV must be within the speed limits. $v_{e_i}$ is the speed on edge $e_i$. $v_{min}(e_i)$ and $v_{max}(e_i)$ is the minimum and maximum speed of edge $e_i$. Thus, the first constraint is formulated as:
\begin{equation*}
\forall_{i\in1 \dots n-1 }:\;v_{min}(e_i) \leq v_{e_i} \leq v_{max}(e_i)
\end{equation*}
When charging at the $i^{\text{th}}$ vertex, the energy acquired is given by $EA(u_i)$. The energy acquired is calculated as charge rate, $R_{CH}(u_i)$, multiplied by charge time, $CT_{u_i}$:
\begin{equation*}
\forall_{i\in1 \dots n }:\; EA(u_i) = R_{CH}(u_i) \times CT_{u_i}
\end{equation*}
The energy spent passing edge $e_i$ is given by $ES(e_i)$ and is calculated by multiplying the distance $D(e_i)$ by the consumption rate of the vehicle $R_{CO}(v_{e_i}))$ at speed $v_{e_i}$.
\begin{equation*}
\forall_{i\in1 \dots n-1 }:\; ES(e_i) = D(e_i) \times R_{CO}(v_{e_i})
\end{equation*}
Furthermore, the battery level of the EV must be between $0$ and $B_{cap}$.
This constraint is split into two constraints. The first constraint ensures that the EV can not pass an edge without having the required energy. The second constraint ensures that the EV can not overcharge at any charging station. 

For the first constraint: The sum of energy acquired at the charging stations prior to the current edge, must be larger or equal to the sum of energy spent on the edges prior to the current edge and the energy needed to pass the current edge. As shown in Figure \ref{fig:pathexample}, vertex $i$ is prior to edge $i$. $B_{cur} + \sum_{j=1}^{i} EA(u_j)$ is the amount of energy acquired on the path at the $i^{\text{th}}$ vertex, where $B_{cur}$ in this context is the starting battery level of the EV. $\sum_{j=1}^{i} EA(u_j)$ is the combined energy spent on the path at the $i^{\text{th}}$ edge. Combining these we get the following constraint:
\begin{equation*}
\forall_{i\in1 \dots n-1 }:\;0 \leq B_{cur} + \sum_{j=1}^{i} EA(u_j) - \sum_{j=1}^{i} ES(e_j) \leq B_{max}
\end{equation*}
The second constraint is almost exactly like the previous constraint, however it looks one charging station further to ensure the energy does not exceed $B_{cap}$ at any charging station:
\begin{equation*}
\begin{aligned}
\forall_{i\in1 \dots n-1}:\;0 \leq B_{cur} + \sum_{j=1}^{i+1} EA(u_j) - \sum_{j=1}^{i} ES(e_j) \leq B_{max}
\end{aligned}
\end{equation*}
Finally, it should not be possible for the EV to spend a negative amount of time at a charging station:
\begin{equation*}
\forall_{i\in1 \dots n }:\; 0 \leq CT_{u_i}
\end{equation*}
The optimisation problem is a NP-complete problem, as it is a non-linear optimisation problem \cite{Murty1987}. The reason it is non-linear is because
the consumption rate of an EV is a quadratic function and the length divided by speed also is a non-linear function.

\subsection{Optimisation problem}
This section introduces a formal description of the fastest path problem, modelled as an optimization problem.

\begin{figure}[h!]
\centering
    \begin{tikzpicture}[shorten >=1pt,node distance=2cm,>=stealth',thick]
        \node[state] (1) {$u_1$};
        \node[state] (2) [right of=1] {$u_2$};
        \node[] (dots) [right of=2] {$\dots$};
        \node[state] (n) [right of=dots] {$u_n$};
        \node[state,draw=none] (d1) [right of=n] {};
        \draw [->] (1) to[right] node[auto] {$e_1$} (2);
        \draw [->] (2) to[right] node[auto] {$e_2$} (dots);
        \draw [->] (dots) to[right] node[auto] {$e_{n-1}$} (n);
        \draw [->] (n) to[right] node[auto] {$e_n$} (d1);
    \end{tikzpicture}
    \caption{Example of a path consisting of $n$ edges} \label{fig:pathexample}
\end{figure} 


The objective of the optimisation problem is to minimise the time spent driving 
and the time spent charging a path from $s$ to $t$. The total time spend driving a path can be expressed as follows:

\begin{equation}
	\begin{aligned} & 
	{\text{min:}}
	& & \sum_{i=1}^{n} \frac{D(e_i)}{v_{e_i}} + CT_{u_i} \\
	\end{aligned}
\end{equation}\label{eq:objfunction}

Where $D(e_i)$ is the distance of road segment $i$, $v_{e_i}$ is the speed driven on road segment $i$ and $CT_{u_i}$ is the time spent charging at the $i^{\text{\tiny th}}$ node of the path. It should be clear that the objective function of this problem is concerned with time, being that $\frac{D(e_i)}{v_{e_i}}$ is the time spent driving and $CT_{u_i}$ is the time spent charging. The objective function has two unknown variables. The unknown variables of the objective function are constrained by the \emph{physical properties} of the $EV$ driving the path.  

On every edge the speed of the EV must be within the speed limits. $v_{e_i}$ is the speed on edge $e_i$. $v_{min}(e_i)$ and $v_{max}(e_i)$ is the minimum and maximum speed. Thus, the first constraint is formulated as:

\begin{equation}
\forall_{i\in1 \dots n }:\;v_{min}(e_i) \leq v_{e_i} \leq v_{max}(e_i)  
\end{equation}

The battery level of the EV, must be between $0$ and $B_{max}$, the battery capacity of the vehicle. 
This constraint is split in to two constraints. The first constraint ensures that the EV can not pass an edge without having the required energy. The second constraint ensures that the EV can not overcharge at any charge station. 

\begin{equation}
\forall_{i\in1 \dots n }:\; EC(u_i) = R_{CH}(u_i) * CT_{u_i}
\end{equation}

$EC(u_i)$ is the energy acquired at the $i^{\text{\tiny th}}$ node of the path. The energy acquired is calculated as charge speed($R_{CH}(u_i)$) multiplied by charge time($CT_{u_i}$) 

\begin{equation}
\forall_{i\in1 \dots n }:\; ES(e_i) = D(e_i)*R_{CO}(v_{e_i})
\end{equation} 

$ES(e_i)$, the energy spent driving $e_i$, is calculated by multiplying the distance of $e_i$ $(D(e_i))$ by  the consumption rate of the vehicle driving at speed $v_{e_i}$ $(R_{CO}(v_{e_i}))$

To ensure the EV can not pass an edge without having the required energy the sum of energy acquired at the charge stations prior to the edge must be larger then the sum of energy used on the edges prior to the edge plus the energy the edge needs. As shown in \Cref{fig:pathexample} node $i$ is prior to edge $i$ and the path have the same amount of edges and nodes. 
\begin{equation}
\forall_{i\in1 \dots n }:\;0 \leq B_{cur} + \sum_{j=1}^{i} EC(u_j) - ES(e_j) \leq B_{max} 
\end{equation}\label{eq:energyreq}
$B_{cur}$ is the battery level of the EV at the start of the path, $ \sum_{j=1}^{i} EC(u_j)$ is the amount of energy acquired on the $j^{\text{\tiny th}}$ node and all node prior to node $i$ and $\sum_{j=1}^{i} EC(u_j)$ is energy used on edge $j$ and all edges prior to the $j^{\text{\tiny th}}$ edge \\

The constraint ensure the EV is not overcharged at any charge station a constraint looking a lot like equation \ref{eq:energyreq} is formulated, however this constrain looks one charge station further to ensure the energy is balanced before and after passing an edge. 
\begin{equation}
\begin{aligned}
\forall_{i\in1 \dots n-1}:\;0 \leq B_{cur} + \sum_{j=1}^{i+1} EC(u_j) - \sum_{j=1}^{i} ES(e_j) \leq B_{max} 
\end{aligned}
\end{equation}
The values of the constraint is the same as in equation \ref{eq:energyreq}

It should also not be possible for the EV to spend a negative amount of time at a charge station:

\begin{equation}
\forall_{i\in1 \dots n }:\; 0 \leq CT_{u_i} 
\end{equation}



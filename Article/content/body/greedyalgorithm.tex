\subsection{Greedy Algorithm}
By continuously picking the currently fastest path, and keep it updated throughout the graph we can find the fastest path. A locally best choice is to select the minimum of driving using the energy in the battery or charge and drive.
The time spent driving an edge $(u_1, u_2)$ in the graph, when driving and charging is given by the following equation:

\[T(v,(u_1, u_2)) = \frac{D(u_1, u_2)}{v} + \frac{CS(v) * D(u_1, u_2) - B_{cur}}{u_1.preCH_0()}\]

\begin{tikzpicture}
%\begin{axis}[]
%\addplot[draw=red]{(50/x)+((0.01*(x^2)+x+10)*100)/40};
%\end{axis}
\end{tikzpicture}


where $v$ is the speed of the vehicle, $D(u_1, u_2)$ is the distance between $u_1$ and $u_2$,
$CS(v)$ is the consumption rate of the vehicle at the speed $v$, $B_{cur}$ is the current battery of the vehicle and $preCH_0(u_1)$ is the charge rate of best charge station previous to $u_2$, which still can be used with overcharging the battery at that station. The above equation yields a function on the form: $av^2 + bv + d/v + c$, due to the fact that $\textbf{CS(v)}$ is a quadratic function \ref{eq:chargingFunc} and the values: $v, CS(v), u_1.preCH_0()$ all being positive.
A, b, c and d are some constants. Represented in a coordinate system, this becomes a curve as represented in figure X. On the x-axis is the speed of the vehicle and on the y-axis is the time spent.

The turning point of the graph, $v_{opt}(e)$, is the optimal speed for the EV this point is easily solved by finding a tangent line with a slope of 0. If $v_{opt}(e)$ is smaller than $v_{min}(e)$, $v_{min}(e)$ defines the optimal speed on road segment $e$ similarly if $v_{opt}(e)$ is larger than $v_{max}(e)$, $v_{max}(e)$ defines the optimal speed.  

The time spent driving an edge $(u_1, u_2)$ using the energy in the battery can be found by solving $B_{cur} - D(u_1, u_2) * CS(v) = 0$, if $v_{opt}(e)$ is lower than $v_{min}(e)$ the time used driving is set to infinity, since there is not enough energy in the battery the drive from $u_1$ to $u_2$, otherwise $v_{opt}$ is decided in the same way when charging is considered.   

Combining the above in a function $travel_time(charge_stations, (u_1, u_2), curbat)$ we have a way of deciding the time it will take to drive a road segment, while accounting for the need of charging along the path. This function can be used in a modified version of Dijkstra's algorithm where we use time instead of distance. Just like Dijkstra's algorithm we keep track of the fastest path leading to each vertices, where fastest path means the path using less time, furthermore the previous charging stations which are still can be used for charging needs to be tracked, this way we can still charge at a station after leaving it if we do not violate the physical constraints in the system(no over charging). Lastly the algorithm needs to keep track of how much battery the EV have when it arrives at each charging station. Using this we can define a fastest path algorithm in the following way: 

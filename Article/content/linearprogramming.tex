\subsection{Approximating a Globally Optimal Path}\label{sec:LP}
In this section we describe an approach to finding the globally optimal solution to the optimisation problem. The optimisation problem is a non-linear optimisation problem, due to \( D(e_i)/v_{e_i} \) and $R_{CO}(v_{e_i})$ being non-linear functions. The problem can however be reformulated into a linear programming problem by reformulating \( D(e_i)/v_{e_i} \) and $R_{CO}(v_{e_i})$ as piecewise-linear functions. Precisely the problem is modelled as a MIP problem. This section introduces the constraints necessary to solve the problem as a MIP problem and thereby end up with a globally optimal solution.
\todo[inline]{Missing some references, and an explanation of what MIP is}

\subsubsection{Linearisation}
To handle the linearisation we need to introduce three new sets, a set of known variables which is the function of each linear piece, and another set of known variables which is the starting point of each line segment and a set of unknown variables which will be the line segments which produces the best result. The unknown variable is called SL(selected line) and it is two dimensional. The first dimension is the the number of functions to be solved and the second is the amount of pieces the function is split into. This lead to the following constraints.   
Exactly one line segment needs to be selected for all line segments. 
\begin{equation*}
\forall_{i\in1 \dots n }:\; \sum_{j=1}^{m} SL_{i,j} = 1
\end{equation*}
$n$ is the length of the path, $m$ is the amount of lines the functions \( D(e_i)/v_{e_i} \) and $R_{CO}(v_{e_i})$ are split into and $\sum_{j=1}^{m} SL_{i,j} = 1$ ensures that only one line segments can be selected.
Being that exactly one line segment needs to be chosen. In other words, SL needs to be binary values. 
\begin{equation*}
\forall_{i\in1 \dots n, j \in 1 \dots m}: \; \; SL_{i,j} \in{0,1} 
\end{equation*}
The speed of the EV needs to be constrained by the line segment chosen:
\begin{equation*}
\forall_{i\in1 \dots n, j \in 1 \dots m-1}:\; SL_{i,j} * P_{i,j}  \le  v_{j,e_i} \le SL_{i,j}*P_{i,j+1}
\end{equation*}
Here $P$ is a set of starting points for all the line segments, which includes the end point of the previous line segment. $SL_{i,j} * P_{i,j}$ is the minimal speed the EV is allowed to drive on edge $i$ and $SL_{i,j}*P_{i,j+1}$ is the maximal speed, note that the values will either be a valid value or $0$ if the segment is not selected. $P_{i,j+1}$ is the end point of $P_{i,j}$. 
The optimal solution to \( D(e_i)/v_{e_i} \) or $R_{CO}(v_{e_i})$ can now be found by multiplying the slope of the selected line with the selected speed and adding the constant b. As an example, here is how we model $R_{CO}(v_{e_i})$ for $i$
\todo[inline]{hvad er b?}
\begin{equation*}
\sum_{j=1}^{m} LinesA_{i,j}*v_{j,e_i} + \sum_{j=1}^{m} SL_{i,j}*LinesB_{i,j} 
\end{equation*}
$\sum_{j=1}^{m} LinesA_{i,j}*v_{j,e_i}$ is equlent of $a*x$ and $\sum_{j=1}^{m} SL_{i,j}*LinesB_{i,j}$ is equlent of $b$. 
Finally note that this is a way to determine the solution of  \( D(e_i)/v_{e_i} \) or $R_{CO}(v_{e_i})$ and before the solution can be found the line segments and points for each of both of the non linear functions needs to be precomputed. 

%sources
%1, SWP-3587
%2, glpk-sos2



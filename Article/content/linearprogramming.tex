\subsection{Linear Program Approach}
The first solution model we propose for finding the fastest path between two points in a road network is based on a linear programming approach. 
The algorithm consists of continuous pruning of paths in the road network, until we end up with a single path which is the optimal one. Initially, we pick a random path from $s$ to $t$, called $p$, and solve it optimally, which returns time $T_p$. For every other paths, $p'_i$, from $s$ to $t$ in the road network, we calculate the time it takes to pass it at maximum speed and without charging, which returns time $T_{p'_{i}}$. If $T_p \leq T_{p'_{i}}$ for any $p'_{i}$, we simply ignore it and all it's sub-paths. We can do this, since the optimally time spend driving $p'_i$ will be either greater than or equal to $T_{p'_{i}}$. If we find a path which can be driven faster than $T_p$ we update $p$ and $T_{p'}$. Determining how a path is optimally solved is described in section \ref{sec:optimizingwithLP}. But the overall algorithm for finding the fastest path looks as follows:

\begin{algorithmic}
\Function{fastestPath}{$G,s,t$}
    \State $p \gets$ initial\_path($G,s,t$) 
    \State $T_p \gets$ solve\_optimally($p$)
    \State $potential\_paths \gets$ get\_paths($G,s,t,T_p$)
    \Repeat 
    	\State $T_{p_1} \gets$ solve\_optimally(potential\_paths[1])
    	\If{$T_{p_1} < T_p$} 
    		\State $T_p \gets T_{p_1}$
    		\State $p \gets potential\_paths[1]$ 
    	\EndIf  
    	\State $paths.remove(potential\_paths[1])$
    	\ForAll{$p \in potential\_paths$} 
    		\If{$T_p \leq p.min$}
    			\State $potential\_paths.remove(p)$
    		\EndIf
    	\EndFor
    \Until{$potential\_paths.size = 0$}
    \State \Return $(p, T_p)$
\EndFunction
\end{algorithmic}

$initial\_path(G,s,t)$ returns a randomly chosen path, $p$ which is solved optimally by $solve\_optimally(p)$. $get\_paths(G,s,t,T_p)$ returns a sorted list of paths in $G$ between $s$ and $t$ with a max travelling time of $T_p$.

\subsection{Optimization problem}
\todo[inline]{the problem definition should come before the algorithm solution}
We now describe how we calculate the speed to drive at and the amount of energy to charge at which charge stations, in order to pass a path optimally in terms of time. The paths the optimization problem is intended to solve are modelled in the following way: \\
\begin{figure}[h!]
\centering
    \begin{tikzpicture}[shorten >=1pt,node distance=2cm,>=stealth',thick]
        \node[state] (1) {$u_1$};
        \node[state] (2) [right of=1] {$u_2$};
        \node[] (dots) [right of=2] {$\dots$};
        \node[state] (n) [right of=dots] {$u_n$};
        \node[state,draw=none] (d1) [right of=n] {};
        \draw [->] (1) to[right] node[auto] {$e_1$} (2);
        \draw [->] (2) to[right] node[auto] {$e_2$} (dots);
        \draw [->] (dots) to[right] node[auto] {$e_{n-1}$} (n);
        \draw [->] (n) to[right] node[auto] {$e_n$} (d1);
    \end{tikzpicture}
    \caption{Example of a path consisting of $n$ edges} \label{fig:pathexample}
\end{figure} \\

The objective of the optimization problem is to minimize the time spend driving 
and the time spend charging for the entire path. The optimization problem can be expressed as follows:

\begin{equation}
	\begin{aligned} & \underset{v_{e_{1}} \dots v_{e_{n}}, charge\_time(u_1 \dots u_n)}
	{\text{minimize:}}
	& & \sum_{i=1}^{n} \frac{D(e_i)}{v(e_i)} + charge\_time(u_i) \\
	\end{aligned}
\end{equation}\label{eq:objfunction}

The problem needs to be solved accordingly to a set of constraints, the constraints can be formulated at follows: \\
On all edges the speed of the electrical vehicle must be within the speed limits. $v(e_i)$ is the speed on edge $e_i$. $v_{min}(e_i)$ and $v_{max}(e_i)$ is the minimum and maximum speed limit. So we have that $v_{min}(e_i) <= v(e_i) <= v_{max}(e_i)$ for all .
 
\begin{equation}
v_{min}(e_i) <= v(e_i) <= v_{max}(e_i) \text{for} i = 1..n  
\end{equation}

The battery level of the EV, $B_{cur}$, needs to be between $0$ and the max capacity of the vehicles battery. This constraint is split up in two constraints, one ensuring that the EV will have enough energy to drive each edge and one ensuring that the EV can not over charge at any charging station. $C_R(v)$ is the the energy consumption function of the EV.

\begin{equation}
\begin{aligned}
& \forall_{i \in 1 \dots n}: \; 0 \leq \sum_{j=1}^{i} \; chargerate_j*charge_j - \\
&  \sum_{j=1}^{i} \; distance_j*C_R(v_j) \leq maxbattery \\
 \end{aligned}
\end{equation}

\begin{equation}
\begin{aligned}
& \forall_{i \in 1 \dots n-1}: \; 0 \leq \sum_{j=i}^{i+1} chargerate_j*charge_j - \\
&   distance_i*C_R(v_i) \leq maxbattery \\
 \end{aligned}
\end{equation}

It should also not be possible for the EV to spend a negative amount of time at a charge station.

\begin{equation}
\forall_{i \in 1 \dots n}: \; 0 \leq charge_i 
\end{equation}

This optimization problem is almost a linear programming problem. We to however suffer from a non-linear constraint $distance_i*C_R(v_i)$ since the consumption rate $C_R(v)$ of the EV is not linear. The function can however be estimated using linearization. Using linearization we end up with the following linear programming problem.  
 
minimizing the sum of time spend driving and the sum of time spend charging. 

\begin{equation}
\begin{aligned}
 & \underset{speed_{1 \dots n},charge_{1 \dots n}}
{\text{minimize}}
& & \sum_{i=1}^{n} \frac{Distance_i}{speed_i} + charge_i \\
\end{aligned}
\end{equation}\label{eq:objfunction}

subject to: 
\begin{equation}
\forall_{i\in1 \dots n }:\; \sum_{j=1}^{m} Selectedlines[i,j] = 1
\end{equation}

\begin{equation}
\forall_{i\in1 \dots n, j \in 1 \dots m}: \; \;0\leq Selectedlines[i,j] \leq 1
\end{equation}

\begin{equation}
\forall_{i\in1 \dots n, j \in 1 \dots m}:\; speed[i,j] \le Selectedlines[i,j] * Points1[i,j]
\end{equation}

\begin{equation}
\forall_{i\in1 \dots n, j \in 1 \dots m}:\; speed[i,j] \ge Selectedlines[i,j] * Points2[i,j]
\end{equation}

\begin{equation}
\begin{split}
\forall_{k\in1 \dots n}\;:\;0 \le\sum_{i=1}^{k}chargerate[i]*charge[i]\\
-\sum_{i=1}^{k} distance[i](\sum_{j=1}^{m} LinesA[i,j]*speed[i,j]\\
+\sum_{j=1}^{m} Selectedlines[i,j]*LinesB[i,j]) \le maxbattery\;capacity
\end{split}
\end{equation}


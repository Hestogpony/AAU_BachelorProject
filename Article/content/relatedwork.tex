\section{Related Work}\label{sec:relatedwork}
To the best of our knowledge, there is no directly related research to the fastest path problem for electrical vehicles, which is the focus of this article. There is however, a lot of work on the shortest path problem, which the fastest path problem can be modelled as, i.e. the fastest path can be modelled as a shortest path with time as weight. Probably best known amongst all of the shortest path algorithms is Dijkstra's algorithm \cite{dijkstra1959note}, which we will use later in Section \ref{sec:algo}.

The fastest path problem is an instance of the constrained shortest path problem (CSPP)\cite{joksch1966shortest}, because one does not only have to find the shortest path, but also adhere to certain constraints, in this case the battery. Our problem of time-optimal routing with charging can be framed as such a CSPP, but CSPP is known to be NP-complete \cite{Garey:1979:CIG:578533}. Another interesting topic, in the domain of electric vehicles, is working with energy-optimal routing, instead of the fastest path. In  \cite{artmeier2010shortest}, they use weights in a graph to represent energy gain and loss, the objective then becomes to find the path which has the lowest battery loss. This is possible because of the energy recuperation property of EVs.



Lastly, if one is interested in the practical usage of shortest path algorithms for real-world road networks, many navigation services introduce hierarchical substructures to perform the shortest path computation faster. This is known as contraction hierarchies. It is a concept which could also be applied to the approach introduced in this article, to better scale the running time against large road networks. It is however not the main concern of research. It is mostly a speed-up technique for shortest path computation.
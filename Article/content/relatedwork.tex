\section{Related Work}\label{sec:relatedwork}
To the best of our knowledge, there is no directly related research to the fastest path problem for EVs, which is the focus of this paper. There is however a lot of work on the shortest path problem. Probably best known amongst all of the shortest path algorithms is Dijkstra's algorithm \cite{dijkstra1959note}, which we will use later in Section \ref{sec:algo}.

The fastest path problem is an instance of the constrained shortest path problem (CSPP)\cite{joksch1966shortest}, because one does not only have to find the shortest path, but also adhere to certain constraints, in this case the battery. Our problem of time-optimised route planning can be framed as a CSPP, though CSPP is known to be NP-complete \cite{Garey:1979:CIG:578533}. Another interesting topic, in the domain of electric vehicles, is working with energy-optimal route planning instead of time-optimised route planning. \cite{artmeier2010shortest} uses weights in a graph to represent energy gain and loss. The objective then becomes to find the path which has the lowest energy loss. This is possible because of the energy recuperation property of EVs.

Lastly, if one is interested in the practical usage of shortest path algorithms, for real-world road networks, many navigation systems introduce hierarchical substructures to perform the shortest path computation faster. This is known as contraction hierarchies. It is a concept which could also be applied to the approach introduced in this paper, in order to reduce the running time. It is however not the main concern of the research in this paper.
\subsection{Quality Assessment}
We will shortly investigate how well the greedy heuristic algorithm solves a path. We are only able to test how well a path is passed in terms of speed and charging. We can not investigate how good the actual path chosen is, since solving the entire road network using the linear programming solution, is simply too time demanding. 

To test the quality of the path solution, we implemented the LP solution described in Section \ref{sec:optiprob} using an existing LP solver\footnote{The LP tool used is the GNU Linear Programming Kit (GLPK). GLPK is intended for solving large-scale LP problems, mixed integer programming problems \cite{glpk}}. In our implementation we used 10 lines segments for linearization of the function. 

For this quality assessment, 8 experiments were conducted using the same default settings as previously. For each $s$ and $t$ we find a path and the route time using the greedy heuristic algorithm and we find a path and the route time using the naive algorithm, for the same source and destination! We then input both these paths to the LP solver. We then end up with the following average route times:
\begin{table}[!htb]
\begin{tabular}{ p{1cm} p{1.35cm} p{1.35cm} p{1.25cm} p{1.6cm}}
\hline
& Naive & Naive-LP  & Greedy  & Greedy-LP \\
Time & 7.461& 5.684 & 5.238 & 5.228\\
\hline
\end{tabular}
\caption{The average of the results for the quality test. The results are given in hours}
\label{tab:LP}
\end{table}

Clearly the LP solving of a path is better, as shown in Table \ref{tab:LP}. Interestingly there is only a difference of $0.198\%$ in the greedy solution of a path compared with LP. Also, it is interesting to see that solving the path chosen by the naive algorithm, with LP achieves a better time, but \emph{not better} than the greedy route time. The difference between Naive-LP and greedy is $7.845\%$. This means that the path chosen plays a significant role in the overall time.
\subsection{Experiment: Driving Distance}
In this experiment we wish to investigate how well the greedy algorithm chooses and drives paths, with distance as the independent variable. We have set the charging rates to a constant value of $60 \si{kW}$, such that the charging becomes irrelevant. The initial distance is set to $50 \si{\km}$ and thereafter increases by $50 \si{\km}$ until it reaches $550 \si{km}$. This is done 5 times to acquire an average.

We expect that: as the distance the EV has to travel increases, the difference between the route time the greedy algorithm achieves and the route time the naive achieves, will increase in favour of the greedy algorithm. 

\begin{figure}[!htb]
\centering
\begin{tikzpicture}
\begin{axis}[
	legend pos=north west, 
	xlabel=Distance (km), 
	ylabel=Route time (hours)] 
\addplot table [x=driving distance, y=naive-time, col sep=comma] {data/driving_dist.csv};
	\addlegendentry{Naive}
\addplot table [x=driving distance, y=greedy-time, col sep=comma] {data/driving_dist.csv};
	\addlegendentry{Greedy heuristic}
\end{axis}
\end{tikzpicture}
\caption{Time spent driving various distances with fixed charging rate} 
\label{fig:driving_dist}
\end{figure}

The results of the experiment is illustrated in figure \ref{fig:driving_dist}. At the very first mark they achieve the same route time, this might be because no charging is needed and the optimal choice will be to drive the shortest path in terms of distance divided by $v_{max}$. The results of the experiment shows that: as the distance increases, it becomes more and more relevant to drive at a speed relative the battery level. This observation matches our expectations.
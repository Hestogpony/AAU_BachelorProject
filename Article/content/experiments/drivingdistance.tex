\subsection{Experiment: Driving Distance}
In this experiment we wish to investigate how well the greedy heuristic algorithm solves the fastest path problem, with distance as the independent variable. We have set the charge rates to a constant value of $60 \si{kW}$, such that the charging stations selected becomes irrelevant. The initial route distance is set to $50 \si{\km}$ and thereafter increases by $50 \si{\km}$ until it reaches $550 \si{km}$. Each distance is iterated 3 times to acquire an average.

We expect that: as the distance the EV has to travel increases, the difference between the route time the greedy algorithm achieves and the route time the naive achieves, will increase in favour of the greedy algorithm. We expect this because selecting road segments which are optimised according to the charging stations in the path, as the greedy heuristic algorithm does, should result in better route times.

\begin{figure}[!htb]
\centering
\begin{tikzpicture}
\begin{axis}[
	legend pos=north west, 
	xlabel=Distance (km), 
	ylabel=Route time (hours)] 
\addplot table [x=driving distance, y=naive-time, col sep=comma] {data/driving_dist.csv};
	\addlegendentry{Naive}
\addplot table [x=driving distance, y=greedy-time, col sep=comma] {data/driving_dist.csv};
	\addlegendentry{Greedy heuristic}
\end{axis}
\end{tikzpicture}
\caption{Time spent driving various distances with fixed charging rate} 
\label{fig:driving_dist}
\end{figure}

The results of the experiment is illustrated in figure \ref{fig:driving_dist}. At the very first mark the greedy heuristic algorithm and the naive algorithm achieve the same route time. This might be because no charging is needed and the optimal choice will be to drive the shortest path in terms of distance divided by speed. The results of the experiment shows that: as the distance increases, it becomes more and more relevant to drive at a speed optimised relative to the charging station. This observation matches our expectations.
\subsection{Experiment: Driving Distance}

We compare the performance of the greedy heuristic algorithm to the naive algorithm with the driving distance as the changing parameter. The longer distances the EV has to travel the more significant will a good path selection heuristic become. The initial distance is set to $50 \si{\km}$ for the experiment and thereafter increased by $50 \si{\km}$ for each iteration in the experiment. With full battery the EV has a range of approximately $123 \si{\km}$ at max speed ($130 \si{\km\per\hour}$). This means we shouldn't expect much difference between the time the greedy heuristic algorithm uses and the time the naive algorithm uses to pass a path in the first couple of iterations, because no charging stations has to be visited.

The results of the experiment is illustrated in figure \ref{fig:charge_rate}.  

\begin{figure}
\centering
\begin{tikzpicture}
\begin{axis}[
	legend pos=north west, 
	xlabel=Distance (km), 
	ylabel=Path time (hours)] 
\addplot table [x=driving distance, y=naive-time, col sep=comma] {data/driving_dist.csv};
	\addlegendentry{Naive}
\addplot table [x=driving distance, y=greedy-time, col sep=comma] {data/driving_dist.csv};
	\addlegendentry{Greedy heuristic}
\end{axis}
\end{tikzpicture}
\caption{Time spend driving various distances given a charging station density of 20 km} 
\label{fig:driving_dist}
\end{figure}
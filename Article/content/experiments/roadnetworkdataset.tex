\subsection{Road network dataset} 
\label{sub:setup}
To facilitate the experiments, we've had to find a real world road network dataset which contains road distances, speed limits and charging stations of varying charge rates. Such a dataset does not openly exist to our knowledge. Instead we have used OpenStreetMaps which is an open-source collection of map data \footnote{One can read more about OSM at \url{http://www.openstreetmap.org/about}}. OSM has a concept of ways and nodes. Ways represent geographical planar objects e.g.\ roads, cycleways, foot ways etc. A node is a geographical point consisting of a latitude and longitude coordinate. A way is constituted of a set of nodes and some tags which describe meta-information about the way, such as the name of the way and what type of way it is e.g. a road, cycleway, foot way  etc. 

The ways and nodes can easily be converted into a, for us, useful road network structure for experimenting. This is done, by simply filtering away all types of ways accept roads and use these as edges in our road network. To get a notion of speed limits on edges, we derive general speed limits from the type of the roads. OSM carries such information as whether the road is a motorway, residential way, tertiary way etc. The speed limits are set according to Danish speed limits. As for the nodes, we are only interested in creating vertices for the nodes which corresponds to intersections between two roads, all other nodes are ignored. To get a notion of charge rates on the vertices, we have implemented a method which distributes random charge rates on randomly selected vertices in our road network.

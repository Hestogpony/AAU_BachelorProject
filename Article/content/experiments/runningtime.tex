\subsection{Experiment: Running Time}
We want to compare the running time of the greedy heuristic algorithm and the naive algorithm with Dijkstra's algorithm. Instead of the default amount of vertices, we start at 400000 vertices and scale downwards to 100000, in steps of 20000. Below 100000 it is no longer possible to find a $300 \si{\km}$ path in our road network.

Technical details of the experiment environment:\\
All the algorithms were implemented in Python 2.7. The road network is kept in memory and the computation were conducted using a single thread on an Intel i7-2600 CPU with 2 x 1333 MHz DDR3 memory.

\begin{figure}[!htb]
\centering
\begin{tikzpicture}
\begin{semilogyaxis}[
	legend pos=north west,
	xlabel=Problem size (vertices),
	ylabel=Computation time (seconds),
    ymax=2000]
\addplot table [x=nodes, y=dijkstra, col sep=comma] {data/time_complexity.csv};
	\addlegendentry{Dijkstra's}
\addplot table [x=nodes, y=greedy, col sep=comma] {data/time_complexity.csv};
	\addlegendentry{Greedy heuristic}
\addplot table [x=nodes, y=naive, col sep=comma] {data/time_complexity.csv};
	\addlegendentry{Naive}
\end{semilogyaxis}
\end{tikzpicture}
\caption{Time spend computing a 300km route on different input sizes. Logarithmic y-scale is used} 
\label{fig:time_comp}
\end{figure}
The results of this experiment is as we expected. Dijkstra's algorithm uses 370 milliseconds solving the smallest input of 100000 vertices, where the greedy heuristic algorithm spends 25 seconds. This shows us that a significant overhead exists from calculating edge weights and managing a larger amount of data for each vertex. From the analysis of the running time, we also see an expected increase in computation time relative to Dijkstra's. While we increase the input size 4 times, Dijkstra's computation time increased about 4.4 times, where the greedy heuristic algorithm increased about 9 times.

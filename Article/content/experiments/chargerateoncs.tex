\subsection{Experiment: Charge Rate on Charging Stations}

\begin{figure}
\centering
\begin{tikzpicture}
\begin{axis}[
    xlabel=Charge rate (\%),
    ylabel=Path time (hours)]
\addplot table [x=Charge rate scale, y=naive-time, col sep=comma] {data/charge_rate.csv};
    \addlegendentry{Naive}
\addplot table [x=Charge rate scale, y=greedy-time, col sep=comma] {data/charge_rate.csv};
    \addlegendentry{Greedy heuristic}
\end{axis}
\end{tikzpicture}
\caption{Time spend driving a 300km path on different charge rates} 
\label{fig:charge_rate}
\end{figure}

We compared the performance of the greedy heuristic algorithm to the naive algorithm with the charge rate of charging stations as variable parameter. The minimum distance between charging stations was set to 30 km, the distance of the patch was set to 300 km. The initial charge rates in the road network were evenly distributed between $10 \si{\kW}$ and $100 \si{\kW}$. For each iteration all charge rates are scaled with a constant factor resulting in worse of better charge rates on all charging stations. The resulting graph of the experiment is seen in Figure \ref{fig:charge_rate}. The x-axis displays the charge rate in percent and the y-axis displayes the time it takes to travel the route.

In Figure \ref{fig:charge_rate} it is quite clear that the behaviour of the naive scales quite polynomially with the increasingly better charge stations, however the greedy algorithm seems to decrease linearly with better charge stations. At the point of 5\% there seems to be a big jump, for the greedy algorithm, this suggests that as charging stations get better, new paths are opened up, because they are more attractive. This is a very nice result. 
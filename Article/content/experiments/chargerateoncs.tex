\subsection{Experiment: Charge Rates}

In this experiment we wish to investigate how well the greedy algorithm drives based on the charging rate of the charging stations. Thus, the charging rate of charge stations is the independent variable. As explained previously, the default charging rates are generated in a range of $10-100 \si{kWh}$, in this experiment we multiply all the charging rates with a scaling factor, which starts at $-40\%$, increments by $5\%$ and stops at $75\%$. This is done 5 times to acquire an average. 

\begin{figure}[!htb]
\centering
\begin{tikzpicture}
\begin{axis}[
    xlabel=Charge rate (\%),
    ylabel=Route time (hours)]
\addplot table [x=Charge rate scale, y=naive-time, col sep=comma] {data/charge_rate.csv};
    \addlegendentry{Naive}
\addplot table [x=Charge rate scale, y=greedy-time, col sep=comma] {data/charge_rate.csv};
    \addlegendentry{Greedy heuristic}
\end{axis}
\end{tikzpicture}
\caption{Time spend driving a 300 km path on different charge rates} 
\label{fig:charge_rate}
\end{figure}

In Figure \ref{fig:charge_rate} it is quite clear that the behaviour of the naive algorithm scales quite polynomially with the increasingly better charge stations, however the greedy algorithm seems to decrease linearly with better charge stations. It is also quite clear that the charging goes from being the bottleneck to driving being the bottleneck. At mark $-20\%$ there is a sudden increase in route time for the greedy algorithm, due to one of the experiments increasing the average substantially. It is hard to explain this situation, but it is probably due to the greedy algorithm making some bad choices. This is a good example of the fact that a chain of local optimal choices, does not yield an optimal solution.
\section{The Body}

\subsection{Shorter Not Equal to Faster}

A shorter path does not necessarily mean a faster path for electrical vehicles. 
This is partly due to the fact that an electrical vehicles uses exponentially more energy 
as it's speed increases but also due to the fact that charge times on charge stations
varies a lot. Driving a longer path with a fast charge station on can therefore turn out to
be a faster choice than driving a shorter path with a slow charge station on. This illustrated 
in ???. In the example we assume our car has a battery capacity of 100 kWh and a energy-usages of
0,4 kWh/km. Path 1 consist of two edges with distance = 250 km and speed limit = 100 km/t
and a charge station with a charge speed of 200 kWh/t. Path 2 consist of two edges with
distance 200 km and speed limit 100 km/t and a charge station with a charge speed of 30 kWh/t.
The total time of each path:

\[\textbf{Path 1:} \text{total\_time} = \text{drive\_time} + \text{charge\_time} = 
				((250 km + 250 km) / 100 km/h) + (100 kWh / 200 kWh/h) = 5,5 h\]

\[\textbf{Path 2:} \text{total\_time} = \text{drive\_time} + \text{charge\_time} = 
				((200 km + 200 km) / 100 km/h) + (60 kWh / 30 kWh/h) = 6 h\]
 
\subsection{The graph model}


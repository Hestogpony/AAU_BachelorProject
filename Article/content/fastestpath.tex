\section{Fastest Path}
As shown in \cref{sec:shorternotfaster} finding the fastest path form vertex $s$ to $t$ in the type of graphs($RN$) from \cref{sec:notation} this article is concerned with does not only depend on the edges of the path but also the vertices. This gives rise to formulating a way to solve a path optimally, since we need a way to determine the actual time it will take to follow this path. In \cref{sub:problem_definition} a simple definition of fastest path is proposed in this section we will describe a fastest path in depth and formulate two approaches to determine how a path is optimally driven.  
    
%Finding a shortest path from vertex $s$ to $t$ can easily be done with an algorithm such %as Dijkstra's algorithm. But as we just showed in \cref{sec:shorternotfaster} a shortest %path does not necessarily imply a faster path. For the fastest path problem, we want to %minimize the time it takes to travel from $s$ to $t$. However, the time it take to pass %an edge cannot be pre-computed or generalized as charge rates of charge stations, %consumption rates and the speed of the vehicle all change the time of an edge. All of %these parameters dictates that there exist no optimal sub-structure to the problem. The %topology of road networks also dictate that the triangle inequality will not always hold. %An optimal solution asks for all simple paths to be solved independently, which is too %complex to run on any country-sized road network in practice. The only approach is thus %to change the problem or the procedure to a solution, where an optimal sub-structure %exist and where the mathematical model of the road network is build to satisfy the %triangle inequality as often as possible, resulting in minimal loss in path-time.\\

%This section presents two approaches for solving the fastest path problem. Both of which %is an expansion on Dijkstra's shortest path algorithm. Both use the greedy property of %Dijkstra's, however the way we produce the time for edges is different for each approach. %One uses a Greedy-Heuristic approach and the other uses a linear programming approach.
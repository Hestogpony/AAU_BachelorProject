\section{Experiments}
\label{sec:experiments}
This section presents the experiments for the greedy heuristic algorithm. Our linear programming solution was implemented with GLPK as a test, but the practical usage of it was not found plausible, due to overwhelming running times. Further experiments were therefore omitted. We compare the performance of the greedy heuristic to a naive algorithm, which is presented in section \ref{sub:naivealgorithm}, on five experiments. The experiments have been selected to illustrate the effects from variables in the route planning problem, including: route distance to drive, charge rate on charging stations, consumption rate of the EV, density of charging stations in the road network and the input size.

To show the effect of the different parameters, we use a base case for all the experiments. The base case is as follows:
\begin{itemize}

     \item Route distance: 300km
     \item Charge rates: 10-100kw
     \item EV consumption rate: Tesla model S\cite{teslacon}
     \item EV battery capacity: 50kW (range of 200km at 80km/h)
	 \item EV start battery charge: full
	 \item Input size: 400000 vertices (Denmark)
     \item Density of charging stations in the road network: 1227:1 (at least 20km distance)
 \end{itemize} 

The route distance, consumption rate and battery capacity is chosen to force the EV to charge in most cases. The charging station density is chosen to have a reliable infrastructure, with an even spread of charging stations. In the base case we have 326 stations, when in fact Denmark have more than 4000\cite{Globalevoutlook}. The average charge rate is presumably a lot lower in reality, as very few charging stations are fast chargers.

\subsection{Road network dataset} 
\label{sub:setup}
To facilitate the experiments, we've had to find a real world road network dataset which contains road distances, speed limits and charging stations of varying charge rates. Such a dataset does not openly exist to our knowledge. Instead we have used OpenStreetMaps (OSM) which is an open-source collection of map data \footnote{One can read more about OSM at \url{http://www.openstreetmap.org/about}}. OSM has a concept of ways and nodes. Ways represent geographical planar objects e.g.\ roads, cycleways, foot ways etc. A node is a geographical point consisting of a latitude and longitude coordinate. A way is constituted of a set of nodes and some tags which describe meta-information about the way, such as the name of the way and what type of way it is e.g. a road, cycleway, foot way  etc. 

The ways and nodes can easily be converted into a, for us, useful road network structure for experimenting. This is done, by simply filtering away all types of ways accept roads and use these as edges in our road network. To get a notion of speed limits on edges, we derive general speed limits from the type of the roads. OSM carries such information as whether the road is a motorway, residential way, tertiary way etc. The speed limits are set according to Danish speed limits. As for the nodes, we are only interested in creating vertices for the nodes which corresponds to intersections between two roads, all other nodes are ignored. To get a notion of charge rates on the vertices, we have implemented a method which distributes random charge rates on randomly selected vertices in our road network.


\subsection{Naive algorithm}
\label{sub:naivealgorithm}
We have formulated a naive algorithm for comparison with our greedy heuristic algorithm. The naive algorithm should simulate the way a naive electric vehicle driver would choose to travel through a road network. The naive algorithm works the following way: It greedily follows the fastest path in terms of distance divided by speed. Whenever the battery reaches 40 \% battery capacity, it starts searching for nearby charging stations in the radius allowed by the 40 \% battery capacity given the vehicle's consumption rate. If no charging stations are available with 40 \% battery capacity, the path from $s$ to $t$ is not solvable for the naive algorithm. If, on the other hand, one or more charging stations are reachable, the algorithm will choose the closest one and charge to 100 \% battery capacity or to the battery capacity needed in order to reach $t$. After charging, the algorithm will resume greedily choosing the fastest roads from the charging station to $t$, in terms of distance divided by speed.

\subsection{Experiment: Driving Distance}

We experiment with the driving distance as the changing parameter. The longer distances the EV has to travel, the more charge stations has to be visited, in order to be able to reach the destination. Therefore, being able to pick efficient charging stations on the path becomes essential. The initial distance is set to $50 \si{\km}$ for the experiment and thereafter increased by $50 \si{\km}$ for each iteration in the experiment. With full battery the EV has a range of approximately $123 \si{\km}$ at max speed ($130 \si{\km\per\hour}$). This means we shouldn't expect much difference between the time the greedy heuristic algorithm uses and the time the naive algorithm uses to pass a path in the first couple of iterations, because no charging stations has to be visited.  

\begin{figure}
\centering
\begin{tikzpicture}
\begin{axis}[
	legend pos=north west,
	xlabel=Distance (km),
	ylabel=Path time (hours)]
\addplot table [x=driving distance, y=naive-time, col sep=comma] {driving_dist.csv};
	\addlegendentry{Naive}
\addplot table [x=driving distance, y=greedy-time, col sep=comma] {driving_dist.csv};
	\addlegendentry{Greedy heuristic}
\end{axis}
\end{tikzpicture}
\caption{Time spend driving various distances given a charging station density of 20 km} 
\label{fig:charge_rate}
\end{figure}

\subsection{Experiment: Charge Rate on Charging Stations}

\begin{figure}
\centering
\begin{tikzpicture}
\begin{axis}[
    xlabel=Charge rate (\%),
    ylabel=Path time (hours)]
\addplot table [x=Charge rate scale, y=naive-time, col sep=comma] {data/charge_rate.csv};
    \addlegendentry{Naive}
\addplot table [x=Charge rate scale, y=greedy-time, col sep=comma] {data/charge_rate.csv};
    \addlegendentry{Greedy heuristic}
\end{axis}
\end{tikzpicture}
\caption{Time spend driving a 300km path on different charge rates} 
\label{fig:charge_rate}
\end{figure}

We compared the performance of the greedy heuristic algorithm to the naive algorithm with the charge rate of charging stations as variable parameter. The minimum distance between charging stations was set to 30 km, the distance of the patch was set to 300 km. The initial charge rates in the road network were evenly distributed between $10 \si{\kW}$ and $100 \si{\kW}$. For each iteration all charge rates are scaled with a constant factor resulting in worse of better charge rates on all charging stations. The resulting graph of the experiment is seen in Figure \ref{fig:charge_rate}. The x-axis displays the charge rate in percent and the y-axis displays the time it takes to travel the route.

\todo[inline]{probably needs to be updated because of new data}
In Figure \ref{fig:charge_rate} it is quite clear that the behaviour of the naive scales quite polynomially with the increasingly better charge stations, however the greedy algorithm seems to decrease linearly with better charge stations. At the point of 5\% there seems to be a big jump, for the greedy algorithm, this suggests that as charging stations get better, new paths are opened up, because they are more attractive. This is a very nice result. 

\subsection{Experiment: Density of Charging Stations}
In this experiment, we wish to find the density of charge station, for which the greedy heuristic algorithm works best, for reference the naive time is included. For this experiment, the density of charging stations are measured in the minimum distance between two charging stations. Initially, the minimum distance is set to 5 km which generates 3029 charging stations in Denmark. Table \ref{table:chargedensity} shows the number of charging stations according to the minimum distance between charging stations in the road network.

\begin{table}[!htb]
\centering
		\begin{tabular}{ p{1.85cm} p{0.67cm} p{0.63cm} p{0.63cm} p{0.63cm} p{0.63cm} p{0.63cm} } \hline
		Radius (km): & 5 & 10 & 20 & 30 & 40 & 50 \\ \hline
		Stations: & 3029 & 827 & 326 & 117 & 76 & 49 \\ \hline 
		\end{tabular}
		\caption{number of charging stations corresponding to the minimum distance between two charging stations}
	\label{table:chargedensity}
	\end{table}

\todo[inline]{Er dette table overhoved relevant?}

\begin{figure}[!htb]
\centering
\begin{tikzpicture}
\begin{axis}[
    xlabel=CS Density km,
    ylabel=Path time (hours)]
\addplot table [x=CS-Density, y=naive-time, col sep=comma] {data/cs_density.csv};
\addlegendentry{Naive}
\addplot table [x=CS-Density, y=greedy-time, col sep=comma] {data/cs_density.csv};
\addlegendentry{Greedy}
\end{axis}
\end{tikzpicture}
\caption{Charging station density experiment results} 
\label{fig:cs_density}
\end{figure}


Five experiments were conducted for the greedy algorithm. Samples were taken in increments of $3\si{km}$. In Figure \ref{fig:cs_density} one can see the results, which is an average of all five experiments, fails were not counted in the average, in total there was $34$ out of $80$, $42.5\%$ fails for the greedy and $0$ for the naive. Interestingly, as we remove more and more charge stations, the greedy algorithm seems to stay relatively constant, with a few jumps here and there. Whereas the naive is very much affected by increasing the amount of charging stations. From the rest of the points we can conclude that the algorithm works mostly with a density of charging stations below $40\si{km}$



\subsection{Test: Running Time}
We want to compare the running time of the greedy heuristic algorithm with Dijkstra's algorithm. We use the established base case except for the input size, which will start at 400000 vertices and scale downwards to 100000. Below 100000 it is no longer possible to find a 300km path in our road network.

Technical details of the experiment environment:\\
Both the greedy heuristic and Dijkstra's algorithm were written in Python 2.7, implemented with a minimum heap for the vertices. The input graph is kept in memory and the computation happened with a single thread on a model i7-2600 CPU.
\begin{figure}[!htb]
\centering
\begin{tikzpicture}
\begin{semilogyaxis}[
	legend pos=north west,
	xlabel=Problem size (vertices),
	ylabel=Computation time (seconds),
    ymax=1000]
\addplot table [x=nodes, y=dijkstra, col sep=comma] {data/time_complexity.csv};
	\addlegendentry{Dijkstra's x10}
\addplot table [x=nodes, y=greedy, col sep=comma] {data/time_complexity.csv};
	\addlegendentry{Greedy heuristic}
\addplot table [x=nodes, y=naive, col sep=comma] {data/time_complexity.csv};
	\addlegendentry{Naive}
\end{semilogyaxis}
\end{tikzpicture}
\caption{Time spend computing a 300km route on different input sizes. Logarithmic y-scale is used} 
\label{fig:time_comp}
\end{figure}

The results of this experiment is as we expected. Dijkstra uses 370 milliseconds solving the smallest input, where the greedy heuristic algorithm spend 25 seconds. This shows us that a significant overhead exists from calculating edge weights and managing a larger amount of data for each vertex. As per the analysis of the running time, we also see an expected increase in computation time relative to Dijkstra's. While we increase the input size 4 times, Dijkstra's computation time increased about 4,4 times, where the greedy heuristic algorithm increased about 9 times.







\section{Experiments}
\label{sec:experiments}
This section presents the experiments for the greedy heuristic algorithm. Our linear programming solution was implemented, such that we can test the quality of the greedy choices. We chose not to use LP along with Dijkstras algorithm, because it was too slow. We compare the performance of the greedy heuristic to a naive algorithm, which is presented in section \ref{sub:naivealgorithm}, on five experiments. The experiments have been selected to illustrate the effects from variables in the route planning problem, including: route distance to drive, charge rate on charging stations, consumption rate of the EV, density of charging stations in the road network and the input size.

The default settings for all the experiments are as follows:
\begin{itemize}
     \item Route distance: $300 \si{km}$
     \item Charge rates: randomly generated between $10-100 \si{kW}$
     \item EV consumption rate: $0.019v^2-0.77v+184.4$ $\si{Wh\per km}$, where $v$ is the speed (corresponds approximately to Tesla's Model S\cite{teslacon})
     \item EV battery capacity: 50 $\si{kWh}$
	 \item EV initial battery: 50 $\si{kWh}$
     \item Density of charging stations: minimum distance of $20 \si{km}$ between each charging station)
 \end{itemize} 

The route distance, consumption rate and battery capacity is chosen to force the EV to charge in most cases. The charging station density is chosen to have a reliable infrastructure, with an even spread of charging stations. In the default settings we have 326 stations, when in fact Denmark have more than 4000\cite{Globalevoutlook}. 

We have used the drivable part of Denmark as a baseline for the experiments. The dataset features:
\begin{itemize}
    \item 483407 vertices
    \item 543482 edges
\end{itemize}

\subsection{Road network dataset} 
\label{sub:setup}
To facilitate the experiments, we've had to find a real world road network dataset which contains road distances, speed limits and charging stations of varying charge rates. Such a dataset does not openly exist to our knowledge. Instead we have used OpenStreetMaps which is an open-source collection of map data \footnote{One can read more about OSM at \url{http://www.openstreetmap.org/about}}. OSM has a concept of ways and nodes. Ways represent geographical planar objects e.g.\ roads, cycleways, foot ways etc. A node is a geographical point consisting of a latitude and longitude coordinate. A way is constituted of a set of nodes and some tags which describe meta-information about the way, such as the name of the way and what type of way it is e.g. a road, cycleway, foot way  etc. From this information we can derive that if way $e_1$ and $e_2$ intersect in node $u$, they will share the node $u$. A node is referred to as either a road intersection or as a charging station if the node has been assigned a charge rate.

The ways and nodes can easily be converted into a, for us, useful road network structure for experimenting. This is done, by simply filtering away all types of ways accept roads and use these as edges in the road network. To get a notion of speed limits on edges, we derive general speed limits from the type of the roads. OSM carries such information as whether the road is a motorway, residential way, tertiary way etc. The speed limits are set according to Danish speed limits. As for the nodes, we are only interested in creating vertices for the nodes which corresponds to intersections between two roads. All other nodes in the road network are ignored. To get a notion of charge rate on the vertices we have implemented a method which distributes random charge rates on randomly selected vertices in our road network.
\todo[inline]{Make sure nodes and vertices are not mixed}



\subsection{Naive algorithm}
\label{sub:naivealgorithm}
We have formulated a naive algorithm for comparison with our greedy heuristic algorithm. The naive algorithm should simulate the way a naive electric vehicle driver would choose to travel through a road network. The naive algorithm works the following way: It greedily follows the fastest path in terms of distance divided by speed. Whenever the battery reaches 40 \% battery capacity, it starts searching for nearby charging stations in the radius allowed by the 40 \% battery capacity given the vehicle's consumption rate. If no charging stations are available with 40 \% battery capacity, the path from $s$ to $t$ is not solvable for the naive algorithm. If, on the other hand, one or more charging stations are reachable, the algorithm will choose the closest one and charge to 100 \% battery capacity or to the battery capacity needed in order to reach $t$. After charging, the algorithm will resume greedily choosing the fastest roads from the charging station to $t$, in terms of distance divided by speed.

\subsection{Experiment: Driving Distance}

\todo[inline]{Revise this entire section so it follows the style of Experiments: Density of Charging Stations}

We compare the performance of the greedy heuristic algorithm to the naive algorithm with the driving distance as the changing parameter. The longer distances the EV has to travel the more significant will a good path selection heuristic become. The initial distance is set to $50 \si{\km}$ for the experiment and thereafter increased by $50 \si{\km}$ for each iteration in the experiment. With full battery the EV has a range of approximately $123 \si{\km}$ at max speed ($130 \si{\km\per\hour}$). This means we shouldn't expect much difference between the time the greedy heuristic algorithm uses and the time the naive algorithm uses to pass a path in the first couple of iterations, because no charging stations has to be visited.

The results of the experiment is illustrated in figure \ref{fig:charge_rate}.  

\begin{figure}
\centering
\begin{tikzpicture}
\begin{axis}[
	legend pos=north west, 
	xlabel=Distance (km), 
	ylabel=Path time (hours)] 
\addplot table [x=driving distance, y=naive-time, col sep=comma] {data/driving_dist.csv};
	\addlegendentry{Naive}
\addplot table [x=driving distance, y=greedy-time, col sep=comma] {data/driving_dist.csv};
	\addlegendentry{Greedy heuristic}
\end{axis}
\end{tikzpicture}
\caption{Time spend driving various distances given a charging station density of 20 km} 
\label{fig:driving_dist}
\end{figure}

\subsection{Experiment: Charge Rates}

In this experiment we wish to investigate how well the greedy algorithm drives based on the charging rate of the charging stations. Thus, the charging rate of charge stations is the independent variable. As explained previously, the default charging rates are generated in a range of $10-100 \si{kWh}$, in this experiment we multiply all the charging rates with a scaling factor, which starts at $-40\%$, increments by $5\%$ and stops at $75\%$. This is done 5 times to acquire an average. 

\begin{figure}[!htb]
\centering
\begin{tikzpicture}
\begin{axis}[
    xlabel=Charge rate (\%),
    ylabel=Route time (hours)]
\addplot table [x=Charge rate scale, y=naive-time, col sep=comma] {data/charge_rate.csv};
    \addlegendentry{Naive}
\addplot table [x=Charge rate scale, y=greedy-time, col sep=comma] {data/charge_rate.csv};
    \addlegendentry{Greedy heuristic}
\end{axis}
\end{tikzpicture}
\caption{Time spend driving a 300 km path on different charge rates} 
\label{fig:charge_rate}
\end{figure}

In Figure \ref{fig:charge_rate} it is quite clear that the behaviour of the naive algorithm scales quite polynomially with the increasingly better charge stations, however the greedy algorithm seems to decrease linearly with better charge stations. It is also quite clear that the charging goes from being the bottleneck to driving being the bottleneck. At mark $-20\%$ there is a sudden increase in route time for the greedy algorithm, due to one of the experiments increasing the average substantially. It is hard to explain this situation, but it is probably due to the greedy algorithm making some bad choices. This is a good example of the fact that a chain of local optimal choices, does not yield an optimal solution.

\subsection{Experiment: Density of Charging Stations}
In this experiment, we wish to find the density of charging stations, for which the greedy heuristic algorithm works best. The independent variable for this experiment is the density of charging stations in the road network. Initially, the minimum distance is set to 5 km which generates 3029 charging stations in road network. Table \ref{table:chargedensity} shows the number of charging stations according to the minimum distance between charging stations in the road network.

\begin{table}[!htb]
\centering
		\begin{tabular}{ p{1.85cm} p{0.67cm} p{0.63cm} p{0.63cm} p{0.63cm} p{0.63cm} p{0.63cm} }
        \hline 
		Radius (km): & 5 & 10 & 20 & 30 & 40 & 50 \\ 
		Stations: & 3029 & 827 & 326 & 117 & 76 & 49 \\
        \hline
		\end{tabular}
		\caption{number of charging stations corresponding to the minimum distance between two charging stations}
	\label{table:chargedensity}
	\end{table}

\begin{figure}[!htb]
\centering
\begin{tikzpicture}
\begin{axis}[
    legend pos=north west, 
    xlabel=CS Density km,
    ylabel=Route time (hours)]
\addplot table [x=CS-Density, y=naive-time, col sep=comma] {data/cs_density.csv};
\addlegendentry{Naive}
\addplot table [x=CS-Density, y=greedy-time, col sep=comma] {data/cs_density.csv};
\addlegendentry{Greedy}
\end{axis}
\end{tikzpicture}
\caption{Time spent driving a $300 \si{\km}$ route with various densities of charging stations} 
\label{fig:cs_density}
\end{figure}

Five experiments were conducted for the greedy heuristic algorithm. Samples were taken in increments of $3\si{km}$. In Figure \ref{fig:cs_density} the results can be seen, which is an average of all five experiments. Fails were not counted in the average. In total $34$ out of $80$ experiments failed for the greedy heuristic algorithm and $0$ for the naive. Interestingly, as we remove more and more charging stations, the greedy heuristic algorithm seems to stay relatively constant. Whereas the naive algorithm is significantly affected by decreasing the amount of charging stations. From the rest of the points, we can conclude that the greedy heuristic algorithm works best with a density of charging stations below $40\si{km}$



\subsection{Test: Running Time}

We compared the run-time of the greedy heuristic algorithm to the run-time of Dijstra's algorithm, with the complexity of the road network as changing parameter. The complexity is measured in the number of vertices on the road network. The more vertices on the road network, the more vertices the two algorithms has to visit before they are able to find a fastest path. The set-up of the experiment: distance to drive was 100 km, the minimum distance between charging stations was 20 km. The initial number of vertices in the road network: 450000 vertices. The number of vertices is then counted down with 10000 for each iteration of the test. The resulting graph is illustrated in (REF!).

\subsection{Greedy Heuristic Quality}
We will shortly investigate how well the greedy heuristic algorithm solves \emph{a path}. We are only able to test how well a path is passed (in terms of speed and charging), we can not investigate how good the actual path chosen is, this is simply too time demanding for the entire road network.

To test this we implemented the LP described in Section \ref{sec:optiprob} using an existing LP solver\footnote{The LP tool used is the GNU Linear Programming Kit (GLPK). GLPK is intended for solving large-scale LP problems, mixed integer programming problems \cite{glpk}}. In our implementation we used 10 lines for linearization of the function. 

For this quality assessment, 8 experiments were conducted using the same default settings as previously. For each $s$ and $t$ we find a path and the route time using the greedy heuristic algorithm and we find a path and the route time using the naive algorithm, for the same source and destination! We then input both these paths to the LP solver. We then end up with the following average route times:
\begin{table}[!htb]
\begin{tabular}{ p{1cm} p{1.35cm} p{1.35cm} p{1.25cm} p{1.6cm}}
\hline
& Naive & Naive-LP  & Greedy  & Greedy-LP \\
Route-time & 7.461& 5.684 & 5.238 & 5.228\\
\hline
\end{tabular}
\caption{The average of the results for the quality test. The results are given in hours}
\label{tab:LP}
\end{table}

Clearly the LP solving of a path is better, as shown in Table \ref{tab:LP}. Interestingly there is only a difference of $0.198\%$ in the greedy solution of a path compared with LP. Also, it is interesting to see that solving the path chosen by the naive algorithm, with LP achieves a better time, but \emph{not better} than the greedy route time. The difference between Naive-LP and greedy is $7.845\%$. This means that the path chosen plays a significant role in the overall time.






